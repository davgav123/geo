\documentclass[a4paper]{article}
\usepackage[OT1, OT2]{fontenc}
\setlength{\textheight}{25cm}
\setlength{\textwidth}{18cm}
\setlength{\topmargin}{-25mm}
\setlength{\hoffset}{-25mm}
\def\zn{,\kern-0.09em,}

\newcommand{\Lat}{\fontencoding{OT1}\fontfamily{cmr}\selectfont}

\begin{document}
\thispagestyle{empty}

\fontfamily{wncyr}
\fontencoding{OT2}\selectfont

\begin{flushleft}
Matematichki fakultet\\
Univerziteta u Beogradu
\end{flushleft}

\bigskip

\begin{center}
\textbf{MOLBA\\
ZA ODOBRAVANJE TEME MASTER RADA
}\end{center}

\bigskip

\begin{flushleft}
Molim da se odobri izrada master rada pod naslovom:
\end{flushleft}

\begin{minipage}{16.5cm}
%%%%%%%%%%%%%%%%%%%%%%%%%%%%%%%%%%%%%%%%%%%%%%%%%%%%%%%%%%%%%%%%%%%%%%%%%%%%%%%
% U donji red upisati naziv master rada umesto teksta: "Naziv master rada"    %
%%%%%%%%%%%%%%%%%%%%%%%%%%%%%%%%%%%%%%%%%%%%%%%%%%%%%%%%%%%%%%%%%%%%%%%%%%%%%%%
\textbf{\textit{\zn Distribuirana obrada geoprostornih podataka''}}
\end{minipage}\\
\rule[4mm]{17.5cm}{.05mm}
\begin{flushleft}
\framebox{
\begin{minipage}[t][10.8cm]{17cm}
%%%%%%%%%%%%%%%%%%%%%%%%%%%%%%%%%%%%%%%%%%%%%%%%%%%%%%%%%%%%%%%%%%%%%%%%%%%%%%%
%  -- unutrasnjost pravougaonika --                                           %
%  Umesto donjeg teksta upisati znacaj i specificni cilj master rada          %
%%%%%%%%%%%%%%%%%%%%%%%%%%%%%%%%%%%%%%%%%%%%%%%%%%%%%%%%%%%%%%%%%%%%%%%%%%%%%%%
\textbf{Znachaj teme i oblasti:}

Danas se generishe velika kolichina podataka. Njihova obrada je bitna iz tog razloga shto podaci imaju veliki broj primena, na primer, razne vrste analitike, pravljenje modela mashinskog uchenja i druge. Obrada se izvrshava коришћењем distribuiranih sistema. To su takvi sistemi za koje vazhi da su sachinjeni od umrezhenih mashina koje su u stanju da se medjusobno koordinishu i da paralelno izvrshavaju posao. Ovi sistemi se koriste za obradu velikih kolichina podataka jer je takve podatke najefikasnije obraditi podelom na delove od kojih se svaki paralelno obradjuje na odvojenoj mashini. U distribuiranoj obradi podataka je zastupljen programski jezik Skala u kome su napisani alati poput Sparka i Kafke, koji su medju најкоришћенијима u toj oblasti.

Geoprostorni podaci su podaci koji predstavljaju lokacije na geografskoj mapi, kao i opis tih lokacija. Imaju veliki broj primena. Pored toga shto doprinose radu aplikacijama koje koriste mape, koristite se za optimalan raspored raznih vrsta infrastrukura, zdravstvenih centara i slichno. Takodje, kombinovanjem sa drugim podacima se mogu koristiti za predvidjanje vremenske prognoze kao i za razne geografske vizualizacije. Jedan primer geoprostornih podataka otvorenog koda je OSM. \\

\textbf{Specifichni cilj rada:}

U okviru rada bic1e prikazan nachin rada distribuiranih sistema i tehnologija koje se koriste za obradu velike kolichine podataka. Takodje, bic1e prikazana Skala aplikacija za rad sa geoprostornim podacima u distribuiranim sistemima koja se mozhe koristiti radi boljeg iskustva korisnika koji koriste geografske mape. \\

\textbf{Ostale bitne informacije:}

Literatura

%\begin{tabular}{|c|c|}
%    \hline
%    \multicolumn{2}{|c|}{\textbf{Uput{}stvo za pisanje nashih slova}} \\
%    \hline\hline
%	ligatura & rezultujuc1i simbol  \\
%    \hline
%    \texttt{{\Lat dj}} & dj \\
%    \hline
%    \texttt{{\Lat Dj}} & Dj \\
%    \hline
%    \texttt{{\Lat zh}} & zh \\
%    \hline
%    \texttt{{\Lat Zh}} & Zh \\
%    \hline
%    \texttt{{\Lat lj}} & lj \\
%    \hline
%    \texttt{{\Lat Lj}} & Lj \\
%    \hline
%    \texttt{{\Lat nj}} & nj \\
%    \hline
%    \texttt{{\Lat Nj}} & Nj \\
%    \hline
%    \texttt{{\Lat c1}} & c1 \\
%    \hline
%    \texttt{{\Lat C1}} & C1 \\
%    \hline
%    \texttt{{\Lat ch}} & ch \\
%    \hline
%    \texttt{{\Lat Ch}} & Ch \\
%    \hline
%    \texttt{{\Lat d2}} & d2 \\
%    \hline
%    \texttt{{\Lat D2}} & D2 \\
%    \hline
%    \texttt{{\Lat sh}} & sh \\
%    \hline
%    \texttt{{\Lat Sh}} & Sh \\
%    \hline
%    \texttt{{\Lat ts}} & ts \\
%    \hline
%    \texttt{{\Lat t\{\}s}} & t{}s \\
%    \hline
%\end{tabular}


\end{minipage}
}
\end{flushleft}
\vspace{1cm}
%%%%%%%%%%%%%%%%%%%%%%%%%%%%%%%%%%%%%%%%%%%%%%%%%%%%%%%%%%%%%%%%%%%%%%%%%%%%%%%
% u donji red uneti:         ime i prezime, broj indeksa i modul studenta     %
%%%%%%%%%%%%%%%%%%%%%%%%%%%%%%%%%%%%%%%%%%%%%%%%%%%%%%%%%%%%%%%%%%%%%%%%%%%%%%%
\makebox[9cm][c]{\textbf{David Gavrilovic1, 1100/2019, Информатика}}
%%%%%%%%%%%%%%%%%%%%%%%%%%%%%%%%%%%%%%%%%%%%%%%%%%%%%%%%%%%%%%%%%%%%%%%%%%%%%%%
% u donji red uneti:               ime i prezime mentora                      %
%%%%%%%%%%%%%%%%%%%%%%%%%%%%%%%%%%%%%%%%%%%%%%%%%%%%%%%%%%%%%%%%%%%%%%%%%%%%%%%
Saglasan mentor \makebox[6cm][c]{\textbf{dr Milena Vujoshevic1 Janichic1}} \\
\rule[4mm]{9cm}{.05mm} \hfill \raisebox{4mm}{\makebox[6cm][l]{.\dotfill.}} \\
\raisebox{1cm}%
[9mm][0mm]{\makebox[10cm][c]{\textit{(ime i prezime studenta, br. indeksa, smer i modul)}}} \\
\makebox[10cm]{ }\\
\vspace{-1cm}\\
\rule[2cm]{6.5cm}{.05mm} \hfill \rule[2cm]{6.5cm}{.05mm}\\
\vspace{-2.4cm}\\
\raisebox{2cm}{\makebox[6.5cm][c]{\textit{(svojeruchni potpis studenta)}}}
\hfill \raisebox{2cm}{\makebox[6.5cm][c]{\textit{(svojeruchni potpis mentora)}}}\\
\vspace{-2cm}\\
%%%%%%%%%%%%%%%%%%%%%%%%%%%%%%%%%%%%%%%%%%%%%%%%%%%%%%%%%%%%%%%%%%%%%%%%%%%%%%%
% u donji red uneti datum podnosenja molbe                                    %
%%%%%%%%%%%%%%%%%%%%%%%%%%%%%%%%%%%%%%%%%%%%%%%%%%%%%%%%%%%%%%%%%%%%%%%%%%%%%%%
\makebox[5.5cm][c]{$<$datum$>$}\makebox[5.5cm]{} Chlanovi komisije\\
%%%%%%%%%%%%%%%%%%%%%%%%%%%%%%%%%%%%%%%%%%%%%%%%%%%%%%%%%%%%%%%%%%%%%%%%%%%%%%%
% POPUNJAVA MENTOR (rucno ili na sledeci nacin):                              %
% u donji red umesto .dotfill. upisati podatke o 1. clanu komisije            %
%%%%%%%%%%%%%%%%%%%%%%%%%%%%%%%%%%%%%%%%%%%%%%%%%%%%%%%%%%%%%%%%%%%%%%%%%%%%%%%
\rule[4mm]{5.5cm}{.05mm}\makebox[5.5cm]{ } 1. \makebox[6cm][l]{dr Mirko Spasic1}\\
\vspace{-8mm}\\
\raisebox{4mm}%
[7mm][0mm]{\makebox[5.5cm][c]{\textit{(datum podnoshenja molbe)}}}\makebox[5.5cm]{ }
%%%%%%%%%%%%%%%%%%%%%%%%%%%%%%%%%%%%%%%%%%%%%%%%%%%%%%%%%%%%%%%%%%%%%%%%%%%%%%%
% POPUNJAVA MENTOR (rucno ili na sledeci nacin):                              %
% u donji red umesto .\dotfill. upisati podatke o 2. clanu komisije           %
%%%%%%%%%%%%%%%%%%%%%%%%%%%%%%%%%%%%%%%%%%%%%%%%%%%%%%%%%%%%%%%%%%%%%%%%%%%%%%%
2. \makebox[6cm][l]{.\dotfill.}\\

\vspace{1cm}


\begin{flushleft}
%%%%%%%%%%%%%%%%%%%%%%%%%%%%%%%%%%%%%%%%%%%%%%%%%%%%%%%%%%%%%%%%%%%%%%%%%%%%%%%
% u donji red upisati                 katedru                                 %
%%%%%%%%%%%%%%%%%%%%%%%%%%%%%%%%%%%%%%%%%%%%%%%%%%%%%%%%%%%%%%%%%%%%%%%%%%%%%%%
Katedra \makebox[9.5cm][l]{za informatiku} je saglasna sa predlozhenom temom.
\vspace{-3mm}
\hspace*{13mm} \rule[2.3cm]{9.5cm}{.05mm}\\
\vspace{-1cm}
%%%%%%%%%%%%%%%%%%%%%%%%%%%%%%%%%%%%%%%%%%%%%%%%%%%%%%%%%%%%%%%%%%%%%%%%%%%%%%%
% POPUNJAVA SEF KATEDRE                                                       %
%%%%%%%%%%%%%%%%%%%%%%%%%%%%%%%%%%%%%%%%%%%%%%%%%%%%%%%%%%%%%%%%%%%%%%%%%%%%%%%
\makebox[6.5cm][c]{} \hfill \makebox[6.5cm][c]{}\\
\rule[4mm]{6.5cm}{.05mm} \hfill \rule[4mm]{6.5cm}{.05mm}\\
\vspace{-5mm}
\makebox[6.5cm][c]{\textit{(shef katedre)}} \hfill \makebox[6.5cm][c]{\textit{(datum odobravanja molbe)}}
\end{flushleft}
\end{document} 
